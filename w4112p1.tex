
\documentclass[11pt,a4paper]{report}
\usepackage[utf8]{inputenc}
\usepackage{amsmath}
\usepackage{amsfonts}
\usepackage{amssymb}
\usepackage{natbib}
\author{Cody De La Vara, Evan Drewry}
\title{w4112p1}
\begin{document}
\maketitle

\section*{Abstract}
\section*{Introduction}
\section*{Main Memory Databases}
\section*{Column-store Databases}

\section*{NoSQL and MapReduce}
NoSQL and MapReduce are technologies that have emerged in the past decade as an answer to the explosive increase in data processing demands that has resulted from the emergence of large data-centric internet companies like Google, Amazon, and Facebook. The goal of these two technologies, therefore, is scalability and availability beyond that achievable by traditional relational database systems. Because of this, both MapReduce and the vast majority of NoSQL database systems add a layer of abstraction above cluster parallelization that provides seamless, automatic scaling and fault tolerance.
\subsection*{NoSQL}
NoSQL is a blanket term (and maybe even a misnomer, depending on who you ask) used to classify database systems that do not conform to the traditional relational model.
\begin{itemize}
\item NoSQL = "Not only" SQL 
\item Departure from relational model
\item Lightweight and scalable
\item Sacrifice consistency for scalability
\item Two main types, document stores (Mongo, ...) and key-value stores (BigTable, Cassandra, ...)
\end{itemize}
\subsubsection*{Document Stores}
\paragraph*{MongoDB}
\paragraph*{CouchDB}
\subsubsection*{Key-Value Stores}
\paragraph*{Cassandra}
\paragraph*{BigTable}

\subsection*{MapReduce}
MapReduce is a simple high-level programming model for processing huge quantities of data in parallel on a cluster. It is powerful because it provides a layer of abstraction over all the complexities of parallelization on a large number of nodes--including  execution scheduling, handling of disk and machine failures, communication between machines, and all partitioning of data among the cluster--while still providing a simple and flexible programming model.\cite{dean2008mapreduce}

The 

MapReduce is also the name Google gave to their widely mimicked implementation of the MapReduce model. The most popular open source implementation is Apache's Hadoop. 
\subsubsection*{Hive}

\subsection*{Compared to traditional relational databases}

\pagebreak
\bibliographystyle{plain}
\bibliography{part3}

\end{document}