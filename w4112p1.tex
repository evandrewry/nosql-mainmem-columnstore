
\documentclass[11pt,a4paper]{report}
\usepackage[utf8]{inputenc}
\usepackage{fullpage}

\usepackage{amsmath}
\usepackage{amsfonts}
\usepackage{amssymb}
\usepackage{natbib}
\author{Cody De La Vara, Evan Drewry}
\title{w4112p1}
\begin{document}
\maketitle
\section*{Introduction}
The relational model, initially proposed by Edgar F. Codd in 1969, has dominated the world of databases.\cite{cod70} The relational model was a revolution in data management, providing developers with a high-level way to describe and query their data. \cite{cod70} As time progressed, hardware characteristics changed dramatically while new types of applications focusing on big data and analytics began to emerge. The change in hardware caused new bottlenecks in database performance to appear, driving researchers to reevaluate the traditional relational-model paradigm as a general solution to all database needs. In particular, the great divide between modern CPU and disk bandwidth has inspired new types of database technologies to emerge that minimize disk latency while taking full advantage of the CPU. Scalability has also become a major concern for databases, pushing researchers to build new types of databases that can keep up with expanding businesses.

This report aims to accomplish three things: 1) to highlight the primary components of each emerging technology, 2) to compare different vendors and their key features, and 3) to provide a critical analysis of each technology and their importance.  By examining each trend, we may get a better understanding of their applicability and improve our ability to make intelligent conclusions about future technologies.

\section*{Main Memory Databases}
\section*{Column-store Databases}

\section*{NoSQL and MapReduce}
NoSQL and MapReduce are technologies that have emerged in the past decade to answer to the explosive increase in data processing demands that has resulted from the emergence of Web 2.0 and large, data-centric internet companies like Google, Amazon, and Facebook.\cite{leavitt2010will} The goal of these technologies, therefore, is availability and horizontal scalability beyond what is achievable with traditional relational database systems.

Because of this, both MapReduce and the vast majority of NoSQL database systems add a layer of abstraction above cluster parallelization that provides seamless, automatic scaling and fault tolerance.
\subsection*{NoSQL}
NoSQL is a blanket term (and maybe even a misnomer, depending on who you ask) used to classify database systems that do not conform to the traditional relational model. It is not even fully agreed upon what the abbreviation even stands for, though the most common interpretations are "Not only SQL" or simply "Not relational."\cite{cattell2011scalable} Sometimes the label is done away with altogether, and replaced by the slightly more accurate "Structured Storage." The term NoSQL was coined because SQL ("Structured Query Language") is tightly coupled with the relational model, and the NoSQL trend represents a departure from the "one-size-fits-all" spirit of SQL and relational databases. It is important to note that this terminology does not prescribe any specific data model, nor even a total rejection of SQL and joins; in fact, there exist many databases that fall under the NoSQL umbrella and also have a SQL-like query language associated with them. 

Though the term NoSQL describes what a data store is \textit{not} rather than what it \textit{is}, there are several prevailing characteristics that are core to these so-called "NoSQL" data stores that differentiate them from other non-traditional database solutions like the main memory databases and column-stores described above (though there do exist both in-memory and column-oriented NoSQL data stores). Because the main motivation behind the NoSQL movement is the lack of scalability present in traditional relational databases, these data stores are most strongly characterized by their ability to horizontally scale simple database operations to millions of users, with application loads distributed across many servers.\cite{strauch2011nosql} Most of the other characteristics that have come to define NoSQL data stores are simply consequences of this primary goal.

\subsubsection{Properties of a  general NoSQL system}

\paragraph{A simple data model}
One of the many reasons traditional relational databases have trouble scaling is the rigid, structured data model that defines them. Because of the one-size-fits-all spirit of the relational model, there is lots of unnecessary overhead  introduces A simple, non relational, model that stores STRUCTURED data rather than providing structure for the data like a 
Lightweight and scalable
avoid unnecessary complexity introduced by one-size-fits-all mentality
RDBMS's are feature-rich and rigid
Three main types: document stores (Mongo, ...), key-value stores (memcached, dynamo, ...), and Extensible Record Stores (or wide column stores, or column-oriented) (BigTable, Cassandra, ...)

\paragraph{Loss of ACID compliance}
Sacrifice consistency for scalability
\subparagraph{CAP-theorem}
\begin{description}
\item[Consistency] c
\item[Availability] foo
\item[Partition Tolerance] blah
\end{description}


\subsubsection*{Document Stores}
\paragraph*{MongoDB}
\paragraph*{CouchDB}
\subsubsection*{Key-Value Stores}
\subsubsection*{Extensible Record Stores}
\paragraph*{Cassandra}
\paragraph*{BigTable}

\subsection*{MapReduce}
MapReduce is a simple high-level programming model for processing huge quantities of data in parallel on a cluster. It is powerful because it provides a layer of abstraction over all the complexities of parallelization on a large number of nodes--including  execution scheduling, handling of disk and machine failures, communication between machines, and all partitioning of data among the cluster--while still providing a simple and flexible programming model.\cite{dean2008mapreduce}

The 

MapReduce is also the name Google gave to their widely mimicked implementation of the MapReduce model. The most popular open source implementation is Apache's Hadoop. 
\subsubsection*{Hive}

\subsection*{Compared to traditional relational databases}


\section{Conclusion}
Database technology has adapted to new trends in business and research that demand scalability, availability and performance over massive databases. As these technologies continue to mature, we can expect to find more hybrid technologies on the frontier that combine the best of in-memory, column-oriented, NoSQL, and MapReduce technologies. We discussed In-Memory Databases which take advantage of the RAM access speeds and CPU processing capabilities to provide uniform data access times. As well, we described the column-store paradigm that organizes data by column on disk, rather than by tuple, and uses heavy compression to maximize disk bandwidth for ad-hoc queries. Finally, we examined NoSQL and MapReduce technologies that provide big data with horizontal scalability and availability using distributed clusters. Among these technologies, NoSQL databases are the most interesting for their emphasis on handling unstructured data at a time when the internet provides a plethora of unstructured data ripe for analysis.
\pagebreak
\bibliographystyle{plain}
\bibliography{part3}

\end{document}