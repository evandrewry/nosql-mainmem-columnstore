
\documentclass[11pt,a4paper]{article}
\usepackage[utf8]{inputenc}
\usepackage{amsmath}
\usepackage{amsfonts}
\usepackage{amssymb}
\usepackage{natbib}
\author{Cody De La Vara, Evan Drewry}
\title{w4112p1}
\begin{document}
\maketitle

\section*{Abstract}
\section*{Introduction}
\section*{Main Memory Databases}
\section*{Column-store Databases}

\section*{NoSQL and MapReduce}
Developed initially to cope with huge volumes of data that traditional relational database systems have trouble handling efficiently
\subsection*{NoSQL}
\begin{itemize}
\item NoSQL = "Not only" SQL 
\item Departure from relational model
\item Lightweight and scalable
\item Sacrifice consistency for scalability
\item Two main types, document stores (Mongo, ...) and key-value stores (BigTable, Cassandra, ...)
\end{itemize}
\subsubsection*{Document Stores}
\paragraph*{MongoDB}
\paragraph*{CouchDB}
\subsubsection*{Key-Value Stores}
\paragraph*{Cassandra}
\paragraph*{BigTable}

\subsection*{MapReduce}
MapReduce is a simple high-level programming model for processing huge quantities of data in parallel. It is powerful because it provides a layer of abstraction over all the complexities of parallelization--including all partitioning of data among the cluster, execution scheduling, handling of disk and machine failures, and communication between machines--while still providing a simple and flexible programming model.\cite{dean2008mapreduce}

The 

MapReduce is also the name of Google's widely mimicked implementation, however the most popular implementation is Apache's open source Hadoop. 
\subsubsection*{Hive}

\subsection*{Compared to traditional relational databases}

\pagebreak
\bibliographystyle{plain}
\bibliography{part3}

\end{document}